
% Default to the notebook output style

    


% Inherit from the specified cell style.




    
\documentclass[11pt]{article}

    
    
    \usepackage[T1]{fontenc}
    % Nicer default font (+ math font) than Computer Modern for most use cases
    \usepackage{mathpazo}

    % Basic figure setup, for now with no caption control since it's done
    % automatically by Pandoc (which extracts ![](path) syntax from Markdown).
    \usepackage{graphicx}
    % We will generate all images so they have a width \maxwidth. This means
    % that they will get their normal width if they fit onto the page, but
    % are scaled down if they would overflow the margins.
    \makeatletter
    \def\maxwidth{\ifdim\Gin@nat@width>\linewidth\linewidth
    \else\Gin@nat@width\fi}
    \makeatother
    \let\Oldincludegraphics\includegraphics
    % Set max figure width to be 80% of text width, for now hardcoded.
    \renewcommand{\includegraphics}[1]{\Oldincludegraphics[width=.8\maxwidth]{#1}}
    % Ensure that by default, figures have no caption (until we provide a
    % proper Figure object with a Caption API and a way to capture that
    % in the conversion process - todo).
    \usepackage{caption}
    \DeclareCaptionLabelFormat{nolabel}{}
    \captionsetup{labelformat=nolabel}

    \usepackage{adjustbox} % Used to constrain images to a maximum size 
    \usepackage{xcolor} % Allow colors to be defined
    \usepackage{enumerate} % Needed for markdown enumerations to work
    \usepackage{geometry} % Used to adjust the document margins
    \usepackage{amsmath} % Equations
    \usepackage{amssymb} % Equations
    \usepackage{textcomp} % defines textquotesingle
    % Hack from http://tex.stackexchange.com/a/47451/13684:
    \AtBeginDocument{%
        \def\PYZsq{\textquotesingle}% Upright quotes in Pygmentized code
    }
    \usepackage{upquote} % Upright quotes for verbatim code
    \usepackage{eurosym} % defines \euro
    \usepackage[mathletters]{ucs} % Extended unicode (utf-8) support
    \usepackage[utf8x]{inputenc} % Allow utf-8 characters in the tex document
    \usepackage{fancyvrb} % verbatim replacement that allows latex
    \usepackage{grffile} % extends the file name processing of package graphics 
                         % to support a larger range 
    % The hyperref package gives us a pdf with properly built
    % internal navigation ('pdf bookmarks' for the table of contents,
    % internal cross-reference links, web links for URLs, etc.)
    \usepackage{hyperref}
    \usepackage{longtable} % longtable support required by pandoc >1.10
    \usepackage{booktabs}  % table support for pandoc > 1.12.2
    \usepackage[inline]{enumitem} % IRkernel/repr support (it uses the enumerate* environment)
    \usepackage[normalem]{ulem} % ulem is needed to support strikethroughs (\sout)
                                % normalem makes italics be italics, not underlines
    

    
    
    % Colors for the hyperref package
    \definecolor{urlcolor}{rgb}{0,.145,.698}
    \definecolor{linkcolor}{rgb}{.71,0.21,0.01}
    \definecolor{citecolor}{rgb}{.12,.54,.11}

    % ANSI colors
    \definecolor{ansi-black}{HTML}{3E424D}
    \definecolor{ansi-black-intense}{HTML}{282C36}
    \definecolor{ansi-red}{HTML}{E75C58}
    \definecolor{ansi-red-intense}{HTML}{B22B31}
    \definecolor{ansi-green}{HTML}{00A250}
    \definecolor{ansi-green-intense}{HTML}{007427}
    \definecolor{ansi-yellow}{HTML}{DDB62B}
    \definecolor{ansi-yellow-intense}{HTML}{B27D12}
    \definecolor{ansi-blue}{HTML}{208FFB}
    \definecolor{ansi-blue-intense}{HTML}{0065CA}
    \definecolor{ansi-magenta}{HTML}{D160C4}
    \definecolor{ansi-magenta-intense}{HTML}{A03196}
    \definecolor{ansi-cyan}{HTML}{60C6C8}
    \definecolor{ansi-cyan-intense}{HTML}{258F8F}
    \definecolor{ansi-white}{HTML}{C5C1B4}
    \definecolor{ansi-white-intense}{HTML}{A1A6B2}

    % commands and environments needed by pandoc snippets
    % extracted from the output of `pandoc -s`
    \providecommand{\tightlist}{%
      \setlength{\itemsep}{0pt}\setlength{\parskip}{0pt}}
    \DefineVerbatimEnvironment{Highlighting}{Verbatim}{commandchars=\\\{\}}
    % Add ',fontsize=\small' for more characters per line
    \newenvironment{Shaded}{}{}
    \newcommand{\KeywordTok}[1]{\textcolor[rgb]{0.00,0.44,0.13}{\textbf{{#1}}}}
    \newcommand{\DataTypeTok}[1]{\textcolor[rgb]{0.56,0.13,0.00}{{#1}}}
    \newcommand{\DecValTok}[1]{\textcolor[rgb]{0.25,0.63,0.44}{{#1}}}
    \newcommand{\BaseNTok}[1]{\textcolor[rgb]{0.25,0.63,0.44}{{#1}}}
    \newcommand{\FloatTok}[1]{\textcolor[rgb]{0.25,0.63,0.44}{{#1}}}
    \newcommand{\CharTok}[1]{\textcolor[rgb]{0.25,0.44,0.63}{{#1}}}
    \newcommand{\StringTok}[1]{\textcolor[rgb]{0.25,0.44,0.63}{{#1}}}
    \newcommand{\CommentTok}[1]{\textcolor[rgb]{0.38,0.63,0.69}{\textit{{#1}}}}
    \newcommand{\OtherTok}[1]{\textcolor[rgb]{0.00,0.44,0.13}{{#1}}}
    \newcommand{\AlertTok}[1]{\textcolor[rgb]{1.00,0.00,0.00}{\textbf{{#1}}}}
    \newcommand{\FunctionTok}[1]{\textcolor[rgb]{0.02,0.16,0.49}{{#1}}}
    \newcommand{\RegionMarkerTok}[1]{{#1}}
    \newcommand{\ErrorTok}[1]{\textcolor[rgb]{1.00,0.00,0.00}{\textbf{{#1}}}}
    \newcommand{\NormalTok}[1]{{#1}}
    
    % Additional commands for more recent versions of Pandoc
    \newcommand{\ConstantTok}[1]{\textcolor[rgb]{0.53,0.00,0.00}{{#1}}}
    \newcommand{\SpecialCharTok}[1]{\textcolor[rgb]{0.25,0.44,0.63}{{#1}}}
    \newcommand{\VerbatimStringTok}[1]{\textcolor[rgb]{0.25,0.44,0.63}{{#1}}}
    \newcommand{\SpecialStringTok}[1]{\textcolor[rgb]{0.73,0.40,0.53}{{#1}}}
    \newcommand{\ImportTok}[1]{{#1}}
    \newcommand{\DocumentationTok}[1]{\textcolor[rgb]{0.73,0.13,0.13}{\textit{{#1}}}}
    \newcommand{\AnnotationTok}[1]{\textcolor[rgb]{0.38,0.63,0.69}{\textbf{\textit{{#1}}}}}
    \newcommand{\CommentVarTok}[1]{\textcolor[rgb]{0.38,0.63,0.69}{\textbf{\textit{{#1}}}}}
    \newcommand{\VariableTok}[1]{\textcolor[rgb]{0.10,0.09,0.49}{{#1}}}
    \newcommand{\ControlFlowTok}[1]{\textcolor[rgb]{0.00,0.44,0.13}{\textbf{{#1}}}}
    \newcommand{\OperatorTok}[1]{\textcolor[rgb]{0.40,0.40,0.40}{{#1}}}
    \newcommand{\BuiltInTok}[1]{{#1}}
    \newcommand{\ExtensionTok}[1]{{#1}}
    \newcommand{\PreprocessorTok}[1]{\textcolor[rgb]{0.74,0.48,0.00}{{#1}}}
    \newcommand{\AttributeTok}[1]{\textcolor[rgb]{0.49,0.56,0.16}{{#1}}}
    \newcommand{\InformationTok}[1]{\textcolor[rgb]{0.38,0.63,0.69}{\textbf{\textit{{#1}}}}}
    \newcommand{\WarningTok}[1]{\textcolor[rgb]{0.38,0.63,0.69}{\textbf{\textit{{#1}}}}}
    
    
    % Define a nice break command that doesn't care if a line doesn't already
    % exist.
    \def\br{\hspace*{\fill} \\* }
    % Math Jax compatability definitions
    \def\gt{>}
    \def\lt{<}
    % Document parameters
    \title{1-Basics Basics - Logical Gates}
    
    
    

    % Pygments definitions
    
\makeatletter
\def\PY@reset{\let\PY@it=\relax \let\PY@bf=\relax%
    \let\PY@ul=\relax \let\PY@tc=\relax%
    \let\PY@bc=\relax \let\PY@ff=\relax}
\def\PY@tok#1{\csname PY@tok@#1\endcsname}
\def\PY@toks#1+{\ifx\relax#1\empty\else%
    \PY@tok{#1}\expandafter\PY@toks\fi}
\def\PY@do#1{\PY@bc{\PY@tc{\PY@ul{%
    \PY@it{\PY@bf{\PY@ff{#1}}}}}}}
\def\PY#1#2{\PY@reset\PY@toks#1+\relax+\PY@do{#2}}

\expandafter\def\csname PY@tok@w\endcsname{\def\PY@tc##1{\textcolor[rgb]{0.73,0.73,0.73}{##1}}}
\expandafter\def\csname PY@tok@c\endcsname{\let\PY@it=\textit\def\PY@tc##1{\textcolor[rgb]{0.25,0.50,0.50}{##1}}}
\expandafter\def\csname PY@tok@cp\endcsname{\def\PY@tc##1{\textcolor[rgb]{0.74,0.48,0.00}{##1}}}
\expandafter\def\csname PY@tok@k\endcsname{\let\PY@bf=\textbf\def\PY@tc##1{\textcolor[rgb]{0.00,0.50,0.00}{##1}}}
\expandafter\def\csname PY@tok@kp\endcsname{\def\PY@tc##1{\textcolor[rgb]{0.00,0.50,0.00}{##1}}}
\expandafter\def\csname PY@tok@kt\endcsname{\def\PY@tc##1{\textcolor[rgb]{0.69,0.00,0.25}{##1}}}
\expandafter\def\csname PY@tok@o\endcsname{\def\PY@tc##1{\textcolor[rgb]{0.40,0.40,0.40}{##1}}}
\expandafter\def\csname PY@tok@ow\endcsname{\let\PY@bf=\textbf\def\PY@tc##1{\textcolor[rgb]{0.67,0.13,1.00}{##1}}}
\expandafter\def\csname PY@tok@nb\endcsname{\def\PY@tc##1{\textcolor[rgb]{0.00,0.50,0.00}{##1}}}
\expandafter\def\csname PY@tok@nf\endcsname{\def\PY@tc##1{\textcolor[rgb]{0.00,0.00,1.00}{##1}}}
\expandafter\def\csname PY@tok@nc\endcsname{\let\PY@bf=\textbf\def\PY@tc##1{\textcolor[rgb]{0.00,0.00,1.00}{##1}}}
\expandafter\def\csname PY@tok@nn\endcsname{\let\PY@bf=\textbf\def\PY@tc##1{\textcolor[rgb]{0.00,0.00,1.00}{##1}}}
\expandafter\def\csname PY@tok@ne\endcsname{\let\PY@bf=\textbf\def\PY@tc##1{\textcolor[rgb]{0.82,0.25,0.23}{##1}}}
\expandafter\def\csname PY@tok@nv\endcsname{\def\PY@tc##1{\textcolor[rgb]{0.10,0.09,0.49}{##1}}}
\expandafter\def\csname PY@tok@no\endcsname{\def\PY@tc##1{\textcolor[rgb]{0.53,0.00,0.00}{##1}}}
\expandafter\def\csname PY@tok@nl\endcsname{\def\PY@tc##1{\textcolor[rgb]{0.63,0.63,0.00}{##1}}}
\expandafter\def\csname PY@tok@ni\endcsname{\let\PY@bf=\textbf\def\PY@tc##1{\textcolor[rgb]{0.60,0.60,0.60}{##1}}}
\expandafter\def\csname PY@tok@na\endcsname{\def\PY@tc##1{\textcolor[rgb]{0.49,0.56,0.16}{##1}}}
\expandafter\def\csname PY@tok@nt\endcsname{\let\PY@bf=\textbf\def\PY@tc##1{\textcolor[rgb]{0.00,0.50,0.00}{##1}}}
\expandafter\def\csname PY@tok@nd\endcsname{\def\PY@tc##1{\textcolor[rgb]{0.67,0.13,1.00}{##1}}}
\expandafter\def\csname PY@tok@s\endcsname{\def\PY@tc##1{\textcolor[rgb]{0.73,0.13,0.13}{##1}}}
\expandafter\def\csname PY@tok@sd\endcsname{\let\PY@it=\textit\def\PY@tc##1{\textcolor[rgb]{0.73,0.13,0.13}{##1}}}
\expandafter\def\csname PY@tok@si\endcsname{\let\PY@bf=\textbf\def\PY@tc##1{\textcolor[rgb]{0.73,0.40,0.53}{##1}}}
\expandafter\def\csname PY@tok@se\endcsname{\let\PY@bf=\textbf\def\PY@tc##1{\textcolor[rgb]{0.73,0.40,0.13}{##1}}}
\expandafter\def\csname PY@tok@sr\endcsname{\def\PY@tc##1{\textcolor[rgb]{0.73,0.40,0.53}{##1}}}
\expandafter\def\csname PY@tok@ss\endcsname{\def\PY@tc##1{\textcolor[rgb]{0.10,0.09,0.49}{##1}}}
\expandafter\def\csname PY@tok@sx\endcsname{\def\PY@tc##1{\textcolor[rgb]{0.00,0.50,0.00}{##1}}}
\expandafter\def\csname PY@tok@m\endcsname{\def\PY@tc##1{\textcolor[rgb]{0.40,0.40,0.40}{##1}}}
\expandafter\def\csname PY@tok@gh\endcsname{\let\PY@bf=\textbf\def\PY@tc##1{\textcolor[rgb]{0.00,0.00,0.50}{##1}}}
\expandafter\def\csname PY@tok@gu\endcsname{\let\PY@bf=\textbf\def\PY@tc##1{\textcolor[rgb]{0.50,0.00,0.50}{##1}}}
\expandafter\def\csname PY@tok@gd\endcsname{\def\PY@tc##1{\textcolor[rgb]{0.63,0.00,0.00}{##1}}}
\expandafter\def\csname PY@tok@gi\endcsname{\def\PY@tc##1{\textcolor[rgb]{0.00,0.63,0.00}{##1}}}
\expandafter\def\csname PY@tok@gr\endcsname{\def\PY@tc##1{\textcolor[rgb]{1.00,0.00,0.00}{##1}}}
\expandafter\def\csname PY@tok@ge\endcsname{\let\PY@it=\textit}
\expandafter\def\csname PY@tok@gs\endcsname{\let\PY@bf=\textbf}
\expandafter\def\csname PY@tok@gp\endcsname{\let\PY@bf=\textbf\def\PY@tc##1{\textcolor[rgb]{0.00,0.00,0.50}{##1}}}
\expandafter\def\csname PY@tok@go\endcsname{\def\PY@tc##1{\textcolor[rgb]{0.53,0.53,0.53}{##1}}}
\expandafter\def\csname PY@tok@gt\endcsname{\def\PY@tc##1{\textcolor[rgb]{0.00,0.27,0.87}{##1}}}
\expandafter\def\csname PY@tok@err\endcsname{\def\PY@bc##1{\setlength{\fboxsep}{0pt}\fcolorbox[rgb]{1.00,0.00,0.00}{1,1,1}{\strut ##1}}}
\expandafter\def\csname PY@tok@kc\endcsname{\let\PY@bf=\textbf\def\PY@tc##1{\textcolor[rgb]{0.00,0.50,0.00}{##1}}}
\expandafter\def\csname PY@tok@kd\endcsname{\let\PY@bf=\textbf\def\PY@tc##1{\textcolor[rgb]{0.00,0.50,0.00}{##1}}}
\expandafter\def\csname PY@tok@kn\endcsname{\let\PY@bf=\textbf\def\PY@tc##1{\textcolor[rgb]{0.00,0.50,0.00}{##1}}}
\expandafter\def\csname PY@tok@kr\endcsname{\let\PY@bf=\textbf\def\PY@tc##1{\textcolor[rgb]{0.00,0.50,0.00}{##1}}}
\expandafter\def\csname PY@tok@bp\endcsname{\def\PY@tc##1{\textcolor[rgb]{0.00,0.50,0.00}{##1}}}
\expandafter\def\csname PY@tok@fm\endcsname{\def\PY@tc##1{\textcolor[rgb]{0.00,0.00,1.00}{##1}}}
\expandafter\def\csname PY@tok@vc\endcsname{\def\PY@tc##1{\textcolor[rgb]{0.10,0.09,0.49}{##1}}}
\expandafter\def\csname PY@tok@vg\endcsname{\def\PY@tc##1{\textcolor[rgb]{0.10,0.09,0.49}{##1}}}
\expandafter\def\csname PY@tok@vi\endcsname{\def\PY@tc##1{\textcolor[rgb]{0.10,0.09,0.49}{##1}}}
\expandafter\def\csname PY@tok@vm\endcsname{\def\PY@tc##1{\textcolor[rgb]{0.10,0.09,0.49}{##1}}}
\expandafter\def\csname PY@tok@sa\endcsname{\def\PY@tc##1{\textcolor[rgb]{0.73,0.13,0.13}{##1}}}
\expandafter\def\csname PY@tok@sb\endcsname{\def\PY@tc##1{\textcolor[rgb]{0.73,0.13,0.13}{##1}}}
\expandafter\def\csname PY@tok@sc\endcsname{\def\PY@tc##1{\textcolor[rgb]{0.73,0.13,0.13}{##1}}}
\expandafter\def\csname PY@tok@dl\endcsname{\def\PY@tc##1{\textcolor[rgb]{0.73,0.13,0.13}{##1}}}
\expandafter\def\csname PY@tok@s2\endcsname{\def\PY@tc##1{\textcolor[rgb]{0.73,0.13,0.13}{##1}}}
\expandafter\def\csname PY@tok@sh\endcsname{\def\PY@tc##1{\textcolor[rgb]{0.73,0.13,0.13}{##1}}}
\expandafter\def\csname PY@tok@s1\endcsname{\def\PY@tc##1{\textcolor[rgb]{0.73,0.13,0.13}{##1}}}
\expandafter\def\csname PY@tok@mb\endcsname{\def\PY@tc##1{\textcolor[rgb]{0.40,0.40,0.40}{##1}}}
\expandafter\def\csname PY@tok@mf\endcsname{\def\PY@tc##1{\textcolor[rgb]{0.40,0.40,0.40}{##1}}}
\expandafter\def\csname PY@tok@mh\endcsname{\def\PY@tc##1{\textcolor[rgb]{0.40,0.40,0.40}{##1}}}
\expandafter\def\csname PY@tok@mi\endcsname{\def\PY@tc##1{\textcolor[rgb]{0.40,0.40,0.40}{##1}}}
\expandafter\def\csname PY@tok@il\endcsname{\def\PY@tc##1{\textcolor[rgb]{0.40,0.40,0.40}{##1}}}
\expandafter\def\csname PY@tok@mo\endcsname{\def\PY@tc##1{\textcolor[rgb]{0.40,0.40,0.40}{##1}}}
\expandafter\def\csname PY@tok@ch\endcsname{\let\PY@it=\textit\def\PY@tc##1{\textcolor[rgb]{0.25,0.50,0.50}{##1}}}
\expandafter\def\csname PY@tok@cm\endcsname{\let\PY@it=\textit\def\PY@tc##1{\textcolor[rgb]{0.25,0.50,0.50}{##1}}}
\expandafter\def\csname PY@tok@cpf\endcsname{\let\PY@it=\textit\def\PY@tc##1{\textcolor[rgb]{0.25,0.50,0.50}{##1}}}
\expandafter\def\csname PY@tok@c1\endcsname{\let\PY@it=\textit\def\PY@tc##1{\textcolor[rgb]{0.25,0.50,0.50}{##1}}}
\expandafter\def\csname PY@tok@cs\endcsname{\let\PY@it=\textit\def\PY@tc##1{\textcolor[rgb]{0.25,0.50,0.50}{##1}}}

\def\PYZbs{\char`\\}
\def\PYZus{\char`\_}
\def\PYZob{\char`\{}
\def\PYZcb{\char`\}}
\def\PYZca{\char`\^}
\def\PYZam{\char`\&}
\def\PYZlt{\char`\<}
\def\PYZgt{\char`\>}
\def\PYZsh{\char`\#}
\def\PYZpc{\char`\%}
\def\PYZdl{\char`\$}
\def\PYZhy{\char`\-}
\def\PYZsq{\char`\'}
\def\PYZdq{\char`\"}
\def\PYZti{\char`\~}
% for compatibility with earlier versions
\def\PYZat{@}
\def\PYZlb{[}
\def\PYZrb{]}
\makeatother


    % Exact colors from NB
    \definecolor{incolor}{rgb}{0.0, 0.0, 0.5}
    \definecolor{outcolor}{rgb}{0.545, 0.0, 0.0}



    
    % Prevent overflowing lines due to hard-to-break entities
    \sloppy 
    % Setup hyperref package
    \hypersetup{
      breaklinks=true,  % so long urls are correctly broken across lines
      colorlinks=true,
      urlcolor=urlcolor,
      linkcolor=linkcolor,
      citecolor=citecolor,
      }
    % Slightly bigger margins than the latex defaults
    
    \geometry{verbose,tmargin=1in,bmargin=1in,lmargin=1in,rmargin=1in}
    
    

    \begin{document}
    
    
    \maketitle
    
    

    
    \hypertarget{neural-network-basics---logical-gates}{%
\section{1. Neural Network Basics - Logical
Gates}\label{neural-network-basics---logical-gates}}

\hypertarget{introduction}{%
\subsection{Introduction}\label{introduction}}

Learning Deep Learning can be a comprehensive task. There exists a wide
range of different frameworks, and the discipline is very heavy on
mathematics. It is therefore pretty challenging to know exactly were to
start.

Several companies, including Microsoft and Google, have made easy-to-use
services for Deep Learning available like for example Microsoft
Cognitive Services. While these are a perfect way to add smartness to
your applications based on Deep Learning, these services are a black box
making it very hard to know how it actually works.

On the other and there exists several frameworks to start making Nerual
Networks from the ground up. Like Keras, PyTorch and Tensorflow. While
these a very heavy framework, there seems to be a lack of literature and
material to teach people how Neural Networks actually functions. One
infamous ``Hello World''-example is the Iris classification problem,
where one is going to determine the spieces of a plant based on a set of
attributes like for example size, colour etc. While this example is
perfect to learn of classification in general works, it is not good for
learning how the Nerual Network in itself is working in order to solve
classification problems.

In this series of papers, I will try to teach Deep Learning from a
different perspective, in order to better understand how the Nerual
Network works.

    \hypertarget{what-is-an-artificial-neural-network}{%
\subsection{What is an Artificial Neural
Network?}\label{what-is-an-artificial-neural-network}}

In order to understand how an Artifical Network works, we first have to
understand what it really is. Both as a concept and from a Computer
Science perspective.

\hypertarget{conceptually}{%
\paragraph{Conceptually}\label{conceptually}}

As a concept, an Artificial Neural Network is trying to model and
emulate the neural network found in animal brains. Which is a vast
network of single neurons, a cell, that has inputs and outputs. Sensory
inputs from the body's senses will pass through this network, and result
in an actions. For example when getting exposed for strong light, the
eyes will automatically close, since there is a circuit for it inside
the network. The more the network is exposed for the same sensory input,
the more ``trained'' it becomes. As the connection between the neurons
is getting stronger, and becomes a more likely path. When speaking of
Artifical Neural Networks, this is the ``weights'' of the inputs.

\hypertarget{from-a-data-science-perspective}{%
\paragraph{From a Data Science
Perspective}\label{from-a-data-science-perspective}}

From a Computer Science perspective, Artificial Neural Networks are very
interesting. If we take a look at Computer Science as a whole,
Artificial Neural Networks is only one of many others network types for
computation.

NAND-networks is maybe the most important one, and it is the fundament
of nearly all computers today. By combining NAND-gates in a network, it
is possible to create any type of logical gates, such as AND, OR, XOR
etc. And such network is then considered \textbf{computationally
universal}. Such property is called \textbf{Turing Completeness}.

It turns out, that an Artificial Neural Network also has the property of
being Turing Complete.

    \hypertarget{making-logical-gates-with-artifical-neural-network}{%
\subsection{Making Logical Gates with Artifical Neural
Network}\label{making-logical-gates-with-artifical-neural-network}}

Whit the NAND-gates in mind, once simple place to start how Neural
Networks works would be to create logical gates. With such example, one
can focus only on the mathematics behind Neural Networks, without
concerning any domain. In this paper we will try to create \textbf{AND},
\textbf{OR} and \textbf{NAND}-gates, by training a very simple Neural
Network with only one node.

    First we need to import the libraries we need:

    \begin{Verbatim}[commandchars=\\\{\}]
{\color{incolor}In [{\color{incolor}1}]:} \PY{k+kn}{import} \PY{n+nn}{numpy} \PY{k}{as} \PY{n+nn}{np}
        \PY{k+kn}{import} \PY{n+nn}{matplotlib}\PY{n+nn}{.}\PY{n+nn}{pyplot} \PY{k}{as} \PY{n+nn}{plt} 
        \PY{k+kn}{import} \PY{n+nn}{pandas} \PY{k}{as} \PY{n+nn}{pd}
        
        \PY{k+kn}{import} \PY{n+nn}{tensorflow} \PY{k}{as} \PY{n+nn}{tf}
        \PY{k+kn}{from} \PY{n+nn}{tensorflow}\PY{n+nn}{.}\PY{n+nn}{keras}\PY{n+nn}{.}\PY{n+nn}{models} \PY{k}{import} \PY{n}{Sequential}
        \PY{k+kn}{from} \PY{n+nn}{tensorflow}\PY{n+nn}{.}\PY{n+nn}{keras}\PY{n+nn}{.}\PY{n+nn}{layers} \PY{k}{import} \PY{n}{Dense}
        \PY{k+kn}{from} \PY{n+nn}{tensorflow}\PY{n+nn}{.}\PY{n+nn}{keras}\PY{n+nn}{.}\PY{n+nn}{optimizers} \PY{k}{import} \PY{n}{Adam}
\end{Verbatim}


    Then let us define 3 dataframes containing the \textbf{AND}, \textbf{OR}
and \textbf{NAND} logic

    \begin{Verbatim}[commandchars=\\\{\}]
{\color{incolor}In [{\color{incolor}2}]:} \PY{c+c1}{\PYZsh{}df\PYZus{}and = pd.DataFrame([[0,0,0], [0,1,0], [1,0,0], [1,1,1]], columns = [\PYZdq{}x0\PYZdq{}, \PYZdq{}x1\PYZdq{}, \PYZdq{}y\PYZdq{}])}
        \PY{n}{np\PYZus{}and} \PY{o}{=} \PY{n}{np}\PY{o}{.}\PY{n}{array}\PY{p}{(}
            \PY{p}{[}\PY{p}{[}\PY{l+m+mi}{0}\PY{p}{,}\PY{l+m+mi}{0}\PY{p}{,}\PY{l+m+mi}{0}\PY{p}{]}\PY{p}{,}
            \PY{p}{[}\PY{l+m+mi}{0}\PY{p}{,}\PY{l+m+mi}{1}\PY{p}{,}\PY{l+m+mi}{0}\PY{p}{]}\PY{p}{,}
            \PY{p}{[}\PY{l+m+mi}{1}\PY{p}{,}\PY{l+m+mi}{0}\PY{p}{,}\PY{l+m+mi}{0}\PY{p}{]}\PY{p}{,}
            \PY{p}{[}\PY{l+m+mi}{1}\PY{p}{,}\PY{l+m+mi}{1}\PY{p}{,}\PY{l+m+mi}{1}\PY{p}{]}\PY{p}{]}
        \PY{p}{)}
        
        \PY{n}{np\PYZus{}or} \PY{o}{=} \PY{n}{np}\PY{o}{.}\PY{n}{array}\PY{p}{(}
            \PY{p}{[}\PY{p}{[}\PY{l+m+mi}{0}\PY{p}{,}\PY{l+m+mi}{0}\PY{p}{,}\PY{l+m+mi}{0}\PY{p}{]}\PY{p}{,}
            \PY{p}{[}\PY{l+m+mi}{0}\PY{p}{,}\PY{l+m+mi}{1}\PY{p}{,}\PY{l+m+mi}{1}\PY{p}{]}\PY{p}{,}
            \PY{p}{[}\PY{l+m+mi}{1}\PY{p}{,}\PY{l+m+mi}{0}\PY{p}{,}\PY{l+m+mi}{1}\PY{p}{]}\PY{p}{,}
            \PY{p}{[}\PY{l+m+mi}{1}\PY{p}{,}\PY{l+m+mi}{1}\PY{p}{,}\PY{l+m+mi}{1}\PY{p}{]}\PY{p}{]}
        \PY{p}{)}
        
        \PY{n}{np\PYZus{}nand} \PY{o}{=} \PY{n}{np}\PY{o}{.}\PY{n}{array}\PY{p}{(}
            \PY{p}{[}\PY{p}{[}\PY{l+m+mi}{0}\PY{p}{,}\PY{l+m+mi}{0}\PY{p}{,}\PY{l+m+mi}{1}\PY{p}{]}\PY{p}{,}
            \PY{p}{[}\PY{l+m+mi}{0}\PY{p}{,}\PY{l+m+mi}{1}\PY{p}{,}\PY{l+m+mi}{1}\PY{p}{]}\PY{p}{,}
            \PY{p}{[}\PY{l+m+mi}{1}\PY{p}{,}\PY{l+m+mi}{0}\PY{p}{,}\PY{l+m+mi}{1}\PY{p}{]}\PY{p}{,}
            \PY{p}{[}\PY{l+m+mi}{1}\PY{p}{,}\PY{l+m+mi}{1}\PY{p}{,}\PY{l+m+mi}{0}\PY{p}{]}\PY{p}{]}
        \PY{p}{)}
        
        \PY{n}{X\PYZus{}and}\PY{p}{,}\PY{n}{y\PYZus{}and} \PY{o}{=} \PY{n}{np\PYZus{}and}\PY{p}{[}\PY{p}{:}\PY{p}{,}\PY{p}{:}\PY{l+m+mi}{2}\PY{p}{]}\PY{p}{,} \PY{n}{np\PYZus{}and}\PY{p}{[}\PY{p}{:}\PY{p}{,}\PY{l+m+mi}{2}\PY{p}{]}
        \PY{n}{X\PYZus{}or}\PY{p}{,}\PY{n}{y\PYZus{}or} \PY{o}{=} \PY{n}{np\PYZus{}or}\PY{p}{[}\PY{p}{:}\PY{p}{,}\PY{p}{:}\PY{l+m+mi}{2}\PY{p}{]}\PY{p}{,} \PY{n}{np\PYZus{}or}\PY{p}{[}\PY{p}{:}\PY{p}{,}\PY{l+m+mi}{2}\PY{p}{]}
        \PY{n}{X\PYZus{}nand}\PY{p}{,}\PY{n}{y\PYZus{}nand} \PY{o}{=} \PY{n}{np\PYZus{}nand}\PY{p}{[}\PY{p}{:}\PY{p}{,}\PY{p}{:}\PY{l+m+mi}{2}\PY{p}{]}\PY{p}{,} \PY{n}{np\PYZus{}nand}\PY{p}{[}\PY{p}{:}\PY{p}{,}\PY{l+m+mi}{2}\PY{p}{]}
\end{Verbatim}


    Then let us scatter plot the data, with the value output value as a
label.

    \begin{Verbatim}[commandchars=\\\{\}]
{\color{incolor}In [{\color{incolor}3}]:} \PY{k}{def} \PY{n+nf}{plot\PYZus{}data}\PY{p}{(}\PY{n}{X}\PY{p}{,}\PY{n}{y}\PY{p}{,} \PY{n}{title}\PY{p}{)}\PY{p}{:}
            \PY{n}{colors} \PY{o}{=} \PY{p}{[}\PY{l+s+s1}{\PYZsq{}}\PY{l+s+s1}{blue}\PY{l+s+s1}{\PYZsq{}} \PY{k}{if} \PY{n}{label} \PY{o}{==} \PY{l+m+mi}{1} \PY{k}{else} \PY{l+s+s1}{\PYZsq{}}\PY{l+s+s1}{red}\PY{l+s+s1}{\PYZsq{}} \PY{k}{for} \PY{n}{label} \PY{o+ow}{in} \PY{n}{y}\PY{p}{]}
            
            \PY{n}{plt}\PY{o}{.}\PY{n}{title}\PY{p}{(}\PY{n}{title}\PY{p}{)}
            \PY{n}{plt}\PY{o}{.}\PY{n}{scatter}\PY{p}{(}\PY{n}{X}\PY{p}{[}\PY{p}{:}\PY{p}{,}\PY{l+m+mi}{0}\PY{p}{]}\PY{p}{,} \PY{n}{X}\PY{p}{[}\PY{p}{:}\PY{p}{,}\PY{l+m+mi}{1}\PY{p}{]}\PY{p}{,} \PY{n}{color}\PY{o}{=}\PY{n}{colors}\PY{p}{)}
            \PY{k}{return} \PY{n}{plt}
        
        \PY{n}{plot\PYZus{}data}\PY{p}{(}\PY{n}{X\PYZus{}and}\PY{p}{,}\PY{n}{y\PYZus{}and}\PY{p}{,} \PY{l+s+s1}{\PYZsq{}}\PY{l+s+s1}{AND}\PY{l+s+s1}{\PYZsq{}}\PY{p}{)}\PY{o}{.}\PY{n}{show}\PY{p}{(}\PY{p}{)}
        \PY{n}{plot\PYZus{}data}\PY{p}{(}\PY{n}{X\PYZus{}or}\PY{p}{,}\PY{n}{y\PYZus{}or}\PY{p}{,} \PY{l+s+s1}{\PYZsq{}}\PY{l+s+s1}{OR}\PY{l+s+s1}{\PYZsq{}}\PY{p}{)}\PY{o}{.}\PY{n}{show}\PY{p}{(}\PY{p}{)}
        \PY{n}{plot\PYZus{}data}\PY{p}{(}\PY{n}{X\PYZus{}nand}\PY{p}{,}\PY{n}{y\PYZus{}nand}\PY{p}{,} \PY{l+s+s1}{\PYZsq{}}\PY{l+s+s1}{NAND}\PY{l+s+s1}{\PYZsq{}}\PY{p}{)}\PY{o}{.}\PY{n}{show}\PY{p}{(}\PY{p}{)}
\end{Verbatim}


    \begin{center}
    \adjustimage{max size={0.9\linewidth}{0.9\paperheight}}{output_8_0.png}
    \end{center}
    { \hspace*{\fill} \\}
    
    \begin{center}
    \adjustimage{max size={0.9\linewidth}{0.9\paperheight}}{output_8_1.png}
    \end{center}
    { \hspace*{\fill} \\}
    
    \begin{center}
    \adjustimage{max size={0.9\linewidth}{0.9\paperheight}}{output_8_2.png}
    \end{center}
    { \hspace*{\fill} \\}
    
    \begin{Verbatim}[commandchars=\\\{\}]
{\color{incolor}In [{\color{incolor}11}]:} \PY{k}{def} \PY{n+nf}{build\PYZus{}model}\PY{p}{(}\PY{n}{X}\PY{p}{,}\PY{n}{y}\PY{p}{)}\PY{p}{:}
             \PY{n}{model} \PY{o}{=} \PY{n}{Sequential}\PY{p}{(}\PY{p}{)}
             \PY{n}{model}\PY{o}{.}\PY{n}{add}\PY{p}{(}\PY{n}{Dense}\PY{p}{(}\PY{l+m+mi}{1}\PY{p}{,} \PY{n}{input\PYZus{}shape}\PY{o}{=}\PY{p}{(}\PY{l+m+mi}{2}\PY{p}{,}\PY{p}{)}\PY{p}{,} \PY{n}{activation}\PY{o}{=}\PY{l+s+s2}{\PYZdq{}}\PY{l+s+s2}{sigmoid}\PY{l+s+s2}{\PYZdq{}}\PY{p}{)}\PY{p}{)}
         
             \PY{n}{model}\PY{o}{.}\PY{n}{compile}\PY{p}{(}\PY{n}{Adam}\PY{p}{(}\PY{n}{lr}\PY{o}{=}\PY{l+m+mf}{0.05}\PY{p}{)}\PY{p}{,} \PY{l+s+s1}{\PYZsq{}}\PY{l+s+s1}{binary\PYZus{}crossentropy}\PY{l+s+s1}{\PYZsq{}}\PY{p}{,} \PY{n}{metrics}\PY{o}{=}\PY{p}{[}\PY{l+s+s1}{\PYZsq{}}\PY{l+s+s1}{accuracy}\PY{l+s+s1}{\PYZsq{}}\PY{p}{]}\PY{p}{)}
             \PY{n}{model}\PY{o}{.}\PY{n}{fit}\PY{p}{(}\PY{n}{X}\PY{p}{[}\PY{p}{:}\PY{p}{,}\PY{p}{:}\PY{l+m+mi}{2}\PY{p}{]}\PY{p}{,} \PY{n}{y}\PY{p}{,} \PY{n}{epochs}\PY{o}{=}\PY{l+m+mi}{1000}\PY{p}{,} \PY{n}{verbose} \PY{o}{=} \PY{l+m+mi}{0}\PY{p}{)}
             \PY{k}{return} \PY{n}{model}
         
         \PY{n}{model\PYZus{}and} \PY{o}{=} \PY{n}{build\PYZus{}model}\PY{p}{(}\PY{n}{X\PYZus{}and}\PY{p}{,} \PY{n}{y\PYZus{}and}\PY{p}{)}
         \PY{n}{pred\PYZus{}and} \PY{o}{=} \PY{n}{model\PYZus{}and}\PY{o}{.}\PY{n}{predict\PYZus{}proba}\PY{p}{(}\PY{n}{X\PYZus{}and}\PY{p}{)}
         
         \PY{n}{model\PYZus{}or} \PY{o}{=} \PY{n}{build\PYZus{}model}\PY{p}{(}\PY{n}{X\PYZus{}or}\PY{p}{,} \PY{n}{y\PYZus{}or}\PY{p}{)}
         \PY{n}{pred\PYZus{}or} \PY{o}{=} \PY{n}{model\PYZus{}or}\PY{o}{.}\PY{n}{predict\PYZus{}proba}\PY{p}{(}\PY{n}{X\PYZus{}or}\PY{p}{)}
         
         \PY{n}{model\PYZus{}nand} \PY{o}{=} \PY{n}{build\PYZus{}model}\PY{p}{(}\PY{n}{X\PYZus{}nand}\PY{p}{,} \PY{n}{y\PYZus{}nand}\PY{p}{)}
         \PY{n}{pred\PYZus{}nand} \PY{o}{=} \PY{n}{model\PYZus{}nand}\PY{o}{.}\PY{n}{predict\PYZus{}proba}\PY{p}{(}\PY{n}{X\PYZus{}nand}\PY{p}{)}
         
         \PY{n+nb}{print}\PY{p}{(}\PY{n}{pred\PYZus{}and}\PY{p}{)}
         \PY{n+nb}{print}\PY{p}{(}\PY{n}{pred\PYZus{}or}\PY{p}{)}
         \PY{n+nb}{print}\PY{p}{(}\PY{n}{pred\PYZus{}nand}\PY{p}{)}
\end{Verbatim}


    \begin{Verbatim}[commandchars=\\\{\}]
[[8.0466270e-06]
 [1.8633515e-02]
 [1.8752605e-02]
 [9.7830617e-01]]
[[0.01183903]
 [0.9966374 ]
 [0.99338865]
 [0.99999976]]
[[0.9999969 ]
 [0.9859302 ]
 [0.98588645]
 [0.01531801]]

    \end{Verbatim}

    \begin{Verbatim}[commandchars=\\\{\}]
{\color{incolor}In [{\color{incolor}10}]:} \PY{k}{def} \PY{n+nf}{plot\PYZus{}decision\PYZus{}boundary}\PY{p}{(}\PY{n}{X}\PY{p}{,} \PY{n}{y}\PY{p}{,} \PY{n}{model}\PY{p}{,} \PY{n}{steps}\PY{o}{=}\PY{l+m+mi}{1000}\PY{p}{,} \PY{n}{cmap}\PY{o}{=}\PY{l+s+s1}{\PYZsq{}}\PY{l+s+s1}{Paired}\PY{l+s+s1}{\PYZsq{}}\PY{p}{)}\PY{p}{:}
             \PY{l+s+sd}{\PYZdq{}\PYZdq{}\PYZdq{}}
         \PY{l+s+sd}{    Function to plot the decision boundary and data points of a model.}
         \PY{l+s+sd}{    Data points are colored based on their actual label.}
         \PY{l+s+sd}{    \PYZdq{}\PYZdq{}\PYZdq{}}
             \PY{n}{cmap} \PY{o}{=} \PY{n}{plt}\PY{o}{.}\PY{n}{get\PYZus{}cmap}\PY{p}{(}\PY{n}{cmap}\PY{p}{)}
         
             \PY{c+c1}{\PYZsh{} Define region of interest by data limits}
             \PY{n}{xmin}\PY{p}{,} \PY{n}{xmax} \PY{o}{=} \PY{n}{X}\PY{p}{[}\PY{p}{:}\PY{p}{,}\PY{l+m+mi}{0}\PY{p}{]}\PY{o}{.}\PY{n}{min}\PY{p}{(}\PY{p}{)} \PY{o}{\PYZhy{}} \PY{l+m+mi}{1}\PY{p}{,} \PY{n}{X}\PY{p}{[}\PY{p}{:}\PY{p}{,}\PY{l+m+mi}{0}\PY{p}{]}\PY{o}{.}\PY{n}{max}\PY{p}{(}\PY{p}{)} \PY{o}{+} \PY{l+m+mi}{1}
             \PY{n}{ymin}\PY{p}{,} \PY{n}{ymax} \PY{o}{=} \PY{n}{X}\PY{p}{[}\PY{p}{:}\PY{p}{,}\PY{l+m+mi}{1}\PY{p}{]}\PY{o}{.}\PY{n}{min}\PY{p}{(}\PY{p}{)} \PY{o}{\PYZhy{}} \PY{l+m+mi}{1}\PY{p}{,} \PY{n}{X}\PY{p}{[}\PY{p}{:}\PY{p}{,}\PY{l+m+mi}{1}\PY{p}{]}\PY{o}{.}\PY{n}{max}\PY{p}{(}\PY{p}{)} \PY{o}{+} \PY{l+m+mi}{1}
             \PY{n}{steps} \PY{o}{=} \PY{l+m+mi}{1000}
             \PY{n}{x\PYZus{}span} \PY{o}{=} \PY{n}{np}\PY{o}{.}\PY{n}{linspace}\PY{p}{(}\PY{n}{xmin}\PY{p}{,} \PY{n}{xmax}\PY{p}{,} \PY{n}{steps}\PY{p}{)}
             \PY{n}{y\PYZus{}span} \PY{o}{=} \PY{n}{np}\PY{o}{.}\PY{n}{linspace}\PY{p}{(}\PY{n}{ymin}\PY{p}{,} \PY{n}{ymax}\PY{p}{,} \PY{n}{steps}\PY{p}{)}
             \PY{n}{xx}\PY{p}{,} \PY{n}{yy} \PY{o}{=} \PY{n}{np}\PY{o}{.}\PY{n}{meshgrid}\PY{p}{(}\PY{n}{x\PYZus{}span}\PY{p}{,} \PY{n}{y\PYZus{}span}\PY{p}{)}
         
             \PY{c+c1}{\PYZsh{} Make predictions across region of interest}
             \PY{n}{labels} \PY{o}{=} \PY{n}{model}\PY{o}{.}\PY{n}{predict}\PY{p}{(}\PY{n}{np}\PY{o}{.}\PY{n}{c\PYZus{}}\PY{p}{[}\PY{n}{xx}\PY{o}{.}\PY{n}{ravel}\PY{p}{(}\PY{p}{)}\PY{p}{,} \PY{n}{yy}\PY{o}{.}\PY{n}{ravel}\PY{p}{(}\PY{p}{)}\PY{p}{]}\PY{p}{)}
         
             \PY{c+c1}{\PYZsh{} Plot decision boundary in region of interest}
             \PY{n}{z} \PY{o}{=} \PY{n}{labels}\PY{o}{.}\PY{n}{reshape}\PY{p}{(}\PY{n}{xx}\PY{o}{.}\PY{n}{shape}\PY{p}{)}
         
             \PY{c+c1}{\PYZsh{}\PYZsh{}fig, ax = subplots()}
             \PY{n}{plt}\PY{o}{.}\PY{n}{contourf}\PY{p}{(}\PY{n}{xx}\PY{p}{,} \PY{n}{yy}\PY{p}{,} \PY{n}{z}\PY{p}{,} \PY{n}{cmap}\PY{o}{=}\PY{n}{cmap}\PY{p}{,} \PY{n}{alpha}\PY{o}{=}\PY{l+m+mf}{0.5}\PY{p}{)}
         
             \PY{c+c1}{\PYZsh{} Get predicted labels on training data and plot}
             \PY{n}{train\PYZus{}labels} \PY{o}{=} \PY{n}{model}\PY{o}{.}\PY{n}{predict}\PY{p}{(}\PY{n}{X}\PY{p}{)}
             \PY{n}{plt}\PY{o}{.}\PY{n}{scatter}\PY{p}{(}\PY{n}{X}\PY{p}{[}\PY{p}{:}\PY{p}{,}\PY{l+m+mi}{0}\PY{p}{]}\PY{p}{,} \PY{n}{X}\PY{p}{[}\PY{p}{:}\PY{p}{,}\PY{l+m+mi}{1}\PY{p}{]}\PY{p}{,} \PY{n}{c}\PY{o}{=}\PY{n}{y}\PY{o}{.}\PY{n}{ravel}\PY{p}{(}\PY{p}{)}\PY{p}{,} \PY{n}{cmap}\PY{o}{=}\PY{n}{cmap}\PY{p}{,} \PY{n}{lw}\PY{o}{=}\PY{l+m+mi}{0}\PY{p}{)}
         
             \PY{k}{return} \PY{n}{plt}
         
         \PY{n}{plot\PYZus{}decision\PYZus{}boundary}\PY{p}{(}\PY{n}{X\PYZus{}and}\PY{p}{,}\PY{n}{y\PYZus{}and}\PY{p}{,} \PY{n}{model\PYZus{}and}\PY{p}{,} \PY{n}{cmap} \PY{o}{=} \PY{l+s+s1}{\PYZsq{}}\PY{l+s+s1}{RdBu}\PY{l+s+s1}{\PYZsq{}}\PY{p}{)}\PY{o}{.}\PY{n}{show}\PY{p}{(}\PY{p}{)}
         \PY{n}{plot\PYZus{}decision\PYZus{}boundary}\PY{p}{(}\PY{n}{X\PYZus{}or}\PY{p}{,}\PY{n}{y\PYZus{}or}\PY{p}{,} \PY{n}{model\PYZus{}or}\PY{p}{,} \PY{n}{cmap} \PY{o}{=} \PY{l+s+s1}{\PYZsq{}}\PY{l+s+s1}{RdBu}\PY{l+s+s1}{\PYZsq{}}\PY{p}{)}\PY{o}{.}\PY{n}{show}\PY{p}{(}\PY{p}{)}
         \PY{n}{plot\PYZus{}decision\PYZus{}boundary}\PY{p}{(}\PY{n}{X\PYZus{}nand}\PY{p}{,}\PY{n}{y\PYZus{}nand}\PY{p}{,} \PY{n}{model\PYZus{}nand}\PY{p}{,} \PY{n}{cmap} \PY{o}{=} \PY{l+s+s1}{\PYZsq{}}\PY{l+s+s1}{RdBu}\PY{l+s+s1}{\PYZsq{}}\PY{p}{)}\PY{o}{.}\PY{n}{show}\PY{p}{(}\PY{p}{)}
\end{Verbatim}


    \begin{center}
    \adjustimage{max size={0.9\linewidth}{0.9\paperheight}}{output_10_0.png}
    \end{center}
    { \hspace*{\fill} \\}
    
    \begin{center}
    \adjustimage{max size={0.9\linewidth}{0.9\paperheight}}{output_10_1.png}
    \end{center}
    { \hspace*{\fill} \\}
    
    \begin{center}
    \adjustimage{max size={0.9\linewidth}{0.9\paperheight}}{output_10_2.png}
    \end{center}
    { \hspace*{\fill} \\}
    

    % Add a bibliography block to the postdoc
    
    
    
    \end{document}
